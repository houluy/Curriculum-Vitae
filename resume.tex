%%%%%%%%%%%%%%%%%%%%%%%%%%%%%%%%%%%%%%%%%
% Freeman Curriculum Vitae
% XeLaTeX Template
% Version 2.0 (19/3/2018)
%
% This template originates from:
% http://www.LaTeXTemplates.com
%
% Authors:
% Vel (vel@LaTeXTemplates.com)
% Alessandro Plasmati
%
% License:
% CC BY-NC-SA 3.0 (http://creativecommons.org/licenses/by-nc-sa/3.0/)
%
%!TEX program = xelatex
% NOTICE: This template must be compiled with XeLaTeX, the line above should
% ensure this happens automatically but if it doesn't you will need to specify 
% XeLaTeX as the engine in your editor or script
% 
%%%%%%%%%%%%%%%%%%%%%%%%%%%%%%%%%%%%%%%%%

%----------------------------------------------------------------------------------------
%	PACKAGES AND OTHER DOCUMENT CONFIGURATIONS
%----------------------------------------------------------------------------------------

\documentclass[10pt]{article} % Font size, can be: 10pt, 11pt or 12pt
\usepackage{xeCJK}
\usepackage{graphicx}
\usepackage{enumitem}

\graphicspath{{logos/}}

\def\CPP{{C\nolinebreak[4]\hspace{-.05em}\raisebox{.4ex}{\tiny\bf ++}}}
%\newfontfamily{\fzblack}{fzlthjt.TTF}
%\newCJKfontfamily{\fzybxs}{fzybxs.ttf}
\newcommand{\email}{houlu8674@bupt.edu.cn}
\newcommand{\website}{https://www.houlu.me}
\newcommand{\github}{https://github.com/houluy}
\newcommand{\linkedin}{https://www.linkedin.com/in/houlu}
\newcommand{\wechat}{guaguade}
\newcommand{\weibosite}{http://weibo.com/lucima}
\newcommand{\weibo}{astroooooo}
\newcommand{\IF}[1]{\textcolor{Maroon}{(IF = #1)}}
\setCJKmainfont[Path=fonts/]{fzlthjt}
\setCJKfamilyfont{cvchsfamilyfont}[Path=fonts/]{fzybxs}
\setCJKfamilyfont{skillfont}[Path=fonts/, Scale=1.2]{fzzyjt}
\setCJKfamilyfont{titlefont}[Path=fonts/, Scale=0.95]{fzlb}
\input{structure.tex} % Include the file that specifies the document structure

% Headers and footers can be added with the \lhead{} \rhead{} \lfoot{} \rfoot{} commands
% Example right footer:
%\rfoot{\color{headings}{\sffamily Last update: \today. Typeset with Xe\LaTeX}}

%----------------------------------------------------------------------------------------

\begin{document}

\begin{paracol}{2} % Begin the multi-column environment
	
	%----------------------------------------------------------------------------------------
	%	NAME AND CURRICULUM VITAE TEXT
	%----------------------------------------------------------------------------------------
	
	\parbox[top][0.12\textheight][c]{\linewidth}{ % Parbox to hold the author name and CV text; fixed height to match the coloured box to the right, centred vertically and full line width
		\vspace{-0.04\textheight} % Reduce whitespace above the parbox to separate it from the main content
		\centering % Centre text
		{\Huge 侯 璐}\faMars\\\medskip % Your name
		{\Huge\color{headings}{\CJKfamily{cvchsfamilyfont} 个人简历}}
	}
	
	%----------------------------------------------------------------------------------------
	%	MAJOR RESEARCH PROJECT
	%----------------------------------------------------------------------------------------
	\section{博士论文}

	{\raggedright\textbf{``面向车联网的移动云网络资源管理与优化研究'',来自``\textit{国家自然科学基金项目}''}\\\medskip}
	针对车联网相关业务,利用凸优化、机器学习等工具,研究在分层云计算网络支持下的资源管理问题。
	
	
	\medskip % Extra whitespace before the next section
	\section{主要研究方向}
	
	%----------------------------------------------------------------------------------------
	%	WORK EXPERIENCE
	%----------------------------------------------------------------------------------------
	\begin{supertabular}{rl}
		\tableentry{物联网云平台}{架构设计、实现、性能优化}{}
		\tableentry{LoRaWAN\texttrademark 系统}{协议分析、架构设计、实现、性能优化}{}
		\tableentry{车联网}{分层车云网络资源管理与优化算法研究}{spaceafter}
	\end{supertabular}
	\section{核心项目经历}
	
	% Blank \workposition command to add another job:
	
	%\workposition{} % Duration
	%{} % FT/PT (full time or part time)
	%{} % Employer
	%{} % Job title
	%{} % Description
	
	% All 5 parameters must be supplied but any can be empty if you don't need them
	
	%------------------------------------------------
	
	\chsworkposition{} % Duration
	{} % FT/PT (full time or part time)
	{\textbf{DSRC}协议栈与应用开发(Sep. 2014 -- June 2015)} % Employer
	{\CJKfamily{titlefont} 开发} % Job title
	{参与开发了车联网IEEE 802.11p通信协议的协议栈,并在ARM开发板上实现了碰撞预警等应用。主要开发语言——C。}
	
	%------------------------------------------------
	
	\chsworkposition{} % Duration
	{} % FT/PT (full time or part time)
	{物联网云平台设计与实现(Sep. 2016 -- Now)} % Employer
	{\CJKfamily{titlefont} 副团队领导人} % 
	{主要负责团队人员、项目进度管理,服务端鉴权认证、亚马逊\faAmazon Alexa服务接入、Elasticsearch搜索服务构建等方面开发工作及后期部署、运维、性能测试等事宜。主要开发语言——Node.js。}  % Description
	
	%------------------------------------------------
	\chsworkposition{} % Duration
	{} % FT/PT (full time or part time)
	{\textbf{LoRaWAN\texttrademark} 系统设计与实现(Oct. 2017 -- Now)} % Employer
	{\CJKfamily{titlefont} 团队领导人} % 
	{领导团队设计实现了基于LoRaWAN\texttrademark 协议的整套云服务系统,包括自适应数据速率(ADR)算法设计等。作为团队领导人,我需要负责管理项目进度、任务分配及汇总。此外,我需要把控技术方向,保证项目处于正确的轨道上。我还需要保证文档紧跟着开发进度在完善。我也负责着一部分服务的开发工作。主要开发语言——Node.js。}  % Description
	
	\chsworkposition{} % Duration
	{} % FT/PT (full time or part time)
	{车联网资源分配算法研究(Sep. 2015 -- Now)} % Employer
	{\CJKfamily{titlefont} 个人研究} % Job title
	{研究性工作,通过凸优化、大数据、机器学习等理论解决车联网业务在云计算网络架构中的资源管理问题。参与了多个自然基金项目及横向项目等。主要研究语言:Python,工具:IPython,TensorFlow等。} % Description
	
	%------------------------------------------------
	
	\vspace{-\baselineskip}\medskip % Standardise the whitespace after this section and before the next (the custom command adds too much otherwise)
	
	
	%----------------------------------------------------------------------------------------
	%	REFERENCES
	%----------------------------------------------------------------------------------------
	
	%\section{个人简介}
	%
	%%\textit{References available on request}
	%
	%%------------------------------------------------
	%
	%% Example \tableentry{} command to add another line:
	%
	%%\tableentry{Heading}{Content}{spaceafter}
	%
	%% All 3 parameters must be supplied but any can be empty if you don't need them
	%% A "spaceafter" value in the third parameter will add some vertical space -- this is to be used between headings
	%
	%%------------------------------------------------
	%
	%\begin{supertabular}{rl} % Start a table with two columns, the table will ensure everything is aligned
	%	
	%	%------------------------------------------------
	%	
	%	\tableentry{}{\textbf{}}{spaceafter}
	%	\tableentry{学位}{}{博士}
	%	%\tableentry{Employer}{\href{http://web.mit.edu/physics/}{Department of Physics}}{}
	%	\tableentry{}{\href{https://web.mit.edu}{\textit{Massachusetts Institute of Technology}}}{spaceafter}
	%	\tableentry{Phone}{+1 (617) 253 1000 x5322 (Work)}{}
	%	\tableentry{Mobile}{+1 (232) 842-3583}{}
	%	
	%	%------------------------------------------------
	%	
	%	\tableentry{}{}{} % Creates some additional whitespace between the references
	%	
	%	%------------------------------------------------
	%	
	%	\tableentry{}{\textbf{Dr. Eli Vance}}{spaceafter}
	%	\tableentry{Position}{Scientist (HL1)}{}
	%	\tableentry{Employer}{\href{http://www.bmrf.us}{Black Mesa Research Facility}}{spaceafter}
	%	\tableentry{Email}{\href{mailto:e.vance@bmrf.us}{e.vance@bmrf.us}}{}
	%	\tableentry{Phone}{+1 (800) 786-1410 x6235 (Work)}{}
	%	\tableentry{Mobile}{+1 (201) 632-3901}{}
	%	
	%	%------------------------------------------------
	%	
	%\end{supertabular}
	
	\medskip % Extra whitespace before the next section
	
	%----------------------------------------------------------------------------------------
	
	\switchcolumn % Switch to the next paracol column
	
	%----------------------------------------------------------------------------------------
	%	COLOURED CONTACT DETAILS BOX
	%----------------------------------------------------------------------------------------
	
	\parbox[top][0.12\textheight][c]{\linewidth}{ % Parbox to hold the colour box; fixed height to match the name/CV text to the left, centred vertically and full line width
		\vspace{-0.04\textheight} % Reduce whitespace above the parbox to separate it from the main content
		\colorbox{shade}{ % Create the coloured box
			\begin{supertabular}{p{0.05\linewidth}|p{0.775\linewidth}} % Start a table with two columns, the table will ensure everything is aligned
				\raisebox{-1pt}{\faHome} & 北京市海淀区西土城路10号 \\ % Address
				\raisebox{-1pt}{\faPhone} & +86-15501081468 \\ % Phone number
				\raisebox{0pt}{\small\faEnvelope} & \href{mailto:\email}{\email} \\ % Email address
				\raisebox{-1pt}{\small\faDesktop} & \href{\website}{\website} \\ % Website
				\raisebox{-1pt}{\faGithub} & \href{\github}{\github} \\ % GitHub profile
				\raisebox{-1pt}{\faLinkedinSquare} & \href{\linkedin}{\linkedin} \\ % LinkedIn profile
				\raisebox{-1pt}{\faWechat} &
				\wechat \\
				\raisebox{-1pt}{\faWeibo} &
				\href{\weibosite}{\weibo} \\
				% See fontawesome.pdf in the fonts folder for all icons you can use
			\end{supertabular}
		}
	}
	
	%----------------------------------------------------------------------------------------
	%	EDUCATION
	%----------------------------------------------------------------------------------------
	
	\section{教育经历} 
	
	% Blank \educationentry{} command to add another degree:
	
	%\educationentry{} % Duration
	%{} % Degree
	%{} % Honours, achievements or distinctions (e.g. first class honours)
	%{} % Department
	%{} % Institution
	
	% All 5 parameters must be supplied but any can be empty if you don't need them
	
	%------------------------------------------------
	
	\begin{supertabular}{rl} % Start a table with two columns, the table will ensure everything is aligned
		
		%------------------------------------------------
		
		\educationentry{2014 -- 2019} % Duration
		{工学博士} % Degree
		{专业:通信工程} % Honours, achievements or distinctions (e.g. first class honours)
		{} % Department
		{北京邮电大学-智能计算与通信实验室} % Institution
		
		%------------------------------------------------
		
		\educationentry{2010 -- 2014} % Duration
		{工学学士} % Degree
		{专业:通信工程} % Honours, achievements or distinctions (e.g. first class honours)
		{} % Department
		{北京邮电大学-信息与通信工程学院} % Institution
		
		%------------------------------------------------
		\educationentry{June 2019} % Duration
		{毕业} % Degree
		{} % Honours, achievements or distinctions (e.g. first class honours)
		{} % Department
		{北京邮电大学} % Institution
	\end{supertabular}
	
	%----------------------------------------------------------------------------------------
	%	AWARDS
	%----------------------------------------------------------------------------------------
	
	\section{获奖情况}
	
	% Example \tableentry{} command to add another line:
	
	%\tableentry{Heading}{Content}{spaceafter}
	
	% All 3 parameters must be supplied but any can be empty if you don't need them
	% A "spaceafter" value in the third parameter will add some vertical space -- this is to be used between headings
	
	%------------------------------------------------
	
	\begin{supertabular}{rl} % Start a table with two columns, the table will ensure everything is aligned
		
		%------------------------------------------------
		
		\tableentry{2017}{国家奖学金}{}
		\tableentry{}{\textit{北京邮电大学智能计算与通信实验室}}{spaceafter}
		
		%------------------------------------------------
		
		\tableentry{2018}{北京邮电大学博士生创新基金资助项目}{}
		\tableentry{}{\textit{北京邮电大学智能计算与通信实验室}}{spaceafter}
		
		%------------------------------------------------
		
		\tableentry{2018}{第二届良飞奖}{}
		\tableentry{}{\textit{北京邮电大学智能计算与通信实验室}}{spaceafter}
		
		%------------------------------------------------
		
	\end{supertabular}
	\section{编程技能} 

	% Example \tableentry{} command to add another line:
	
	%\tableentry{Heading}{Content}{spaceafter}
	
	% All 3 parameters must be supplied but any can be empty if you don't need them
	% A "spaceafter" value in the third parameter will add some vertical space -- this is to be used between headings
	
	%------------------------------------------------
	
	\begin{supertabular}{rl} % Start a table with two columns, the table will ensure everything is aligned
		
		%------------------------------------------------
		
		\tableentry{精通}{Python\includegraphics*[width=0.019\textwidth]{python.png}, Node.js\includegraphics*[width=0.019\textwidth]{nodejs.png}}{spaceafter}
		
		%------------------------------------------------
		
		\tableentry{熟悉}{C, \CPP, MATLAB, \LaTeX, SQL,}{}
		\tableentry{}{shell, HTML}{spaceafter}
		
		%------------------------------------------------
		
		\tableentry{了解}{Go, Rust, lisp, Javascript, Java}{spaceafter}
		
		%------------------------------------------------
		
	\end{supertabular}
	\section{工作技能}
	
	% Example \longformdescription{} command to add another section:
	
	%\longformdescription{Heading}{Description}
	
	%------------------------------------------------
	
	\chsskillentry{\CJKfamily{skillfont} 服务架构设计}{理解架构设计原理性知识。具有一定实践经验。}
	
	\chsskillentry{\CJKfamily{skillfont} \textbf{Python开发能力}}{具有多年Python经验,运营着一个Python技术公众号:它不只是Python。}
	
	\chsskillentry{\CJKfamily{skillfont} \textbf{机器学习}}{熟悉机器学习各类算法。}
	
	\chsskillentry{\CJKfamily{skillfont} \textbf{研究能力}}{具有较强的独立或协作研究能力,善于发现问题,并利用合适的工具解决问题。}
	
	\chsskillentry{\CJKfamily{skillfont} \textbf{团队领导力}}{具有一定的团队领导能力。}
	
	\chsskillentry{\CJKfamily{skillfont} \textbf{写作能力}}{具有极强的文档、报告、记录、PPT写作和排版能力。}
	
	
	\section{计算机技能}
	
	\begin{supertabular}{rl} % Start a table with two columns, the table will ensure everything is aligned
		
		%------------------------------------------------
		
		\tableentry{精通}{Microsoft Office, Git, reStructuredText,}{}
		\tableentry{}{Markdown, Linux, vim}{}
		\tableentry{}{HTTP, MQTT, kafka,}{}
		\tableentry{}{ProtocolBuffer, gRPC, TLS, Locust}{}
		\tableentry{}{Github, StackOverflow, Readthedocs}{}
		\tableentry{}{...}{spaceafter}
		%------------------------------------------------
		
		\tableentry{熟悉}{Elasticsearch, Nginx, HAProxy, Redis,}{}
		\tableentry{}{MySQL, MongoDB, }{}
		\tableentry{}{...}{spaceafter}
		%------------------------------------------------
		
		%------------------------------------------------
		
	\end{supertabular}
	
	%----------------------------------------------------------------------------------------
	%	COMPUTER SKILLS
	%----------------------------------------------------------------------------------------

	
	\section{英语技能}
	
	% Example \tableentry{} command to add another line:
	
	%\tableentry{Heading}{Content}{spaceafter}
	
	% All 3 parameters must be supplied but any can be empty if you don't need them
	% A "spaceafter" value in the third parameter will add some vertical space -- this is to be used between headings
	
	%------------------------------------------------
	具有熟练地\textbf{听、说、读、写}英语的能力
	
	%----------------------------------------------------------------------------------------
	%	SKILLS DESCRIPTION
	%----------------------------------------------------------------------------------------
	
	
	
	%----------------------------------------------------------------------------------------
	%	PUBLICATIONS
	%----------------------------------------------------------------------------------------

		
	\switchcolumn
	\section{预期职位}
	
	\begin{enumerate}[font=\color{headings}\bfseries] % Start a table with two columns, the table will ensure everything is aligned
		
		%------------------------------------------------
		
		\item 物联网相关(智慧城市等)工程或研究性岗位
		
		%------------------------------------------------
		
		\item Python相关研发或项目管理岗位
	\end{enumerate}
	%------------------------------------------------
	\section{已发表论文}
	
	% Example \longformdescription{} command to add another publication:
	
	%\longformpublication{Reference (format this manually as desired)}
	
	%------------------------------------------------
	%\renewcommand{\labelenumi}{\theenumi}
	%\renewcommand{\theenumi}{\roman{enumi}}
	\begin{enumerate}[font=\color{headings}\bfseries]
		
		\item \longformpublication{\textbf{Lu Hou}, Shaohang Zhao, Xiong Xiong, Kan Zheng, Periklis Chatzimisios, M. Shamim Hossain, Wei Xiang, ``Internet of things cloud: architecture and implementation,'' \textit{IEEE Communications Magazine}, vol. 54, no. 12, Dec. 2016, pp. 32-39. \IF{10.435}}
		
		\item \longformpublication{\textbf{Lu Hou}, Shaohang Zhao, Xing Li, Periklis Chatzimisios, Kan Zheng, ``Design and implementation of application programming interface for Internet of things cloud,'' \textit{International Journal of Network Management}, vol. 27, no. 3, June 2016. \IF{1.118}}
		
		\item \longformpublication{\textbf{Lu Hou}, Kan Zheng, Periklis Chatzimisios, Yi Feng, ``A continuous-time Markov decision process-based resource allocation scheme in vehicular cloud for mobile video services,'' \textit{Computer Communications}, vol. 118, Mar. 2018, pp. 140-147. \IF{3.338}}
		
		\item \longformpublication{\textbf{Lu Hou}, Lei Lei, Kan Zheng, ``Design on publish/subscribe message dissemination for vehicular networks with mobile edge computing,'' \textit{2017 IEEE GLOBECOM}, Dec. 2017}
		
		\item \longformpublication{\textbf{Lu Hou}, Lei Lei, Kan Zheng, Xianbin Wang, ``A Q-learning based Proactive Caching Strategy for Non-safety Related Services in Vehicular Networks,'' \textit{IEEE Internet of Things Journal}, \textbf{Under review}}
		
		\item \longformpublication{Kan Zheng, \textbf{Lu Hou}, Hanling Meng, Qiang Zheng, Ning Lu, Lei Lei, ``Soft-defined heterogeneous vehicular network: architecture and challenges,'' \textit{IEEE Network}, vol. 30, no. 4, July 2016, pp. 72-80.\IF{7.23}}
		
		\item \longformpublication{Xiong Xiong, \textbf{Lu Hou}, Kan Zheng, Wei Xiang, M. Shamim Hossain, and Sk Md Mizanur Rahman, ``SMDP-based radio resource allocation scheme in software-defined internet of things networks,'' \textit{IEEE Sensors Journal} vol. 16, no. 20, June 2016, pp. 7304-7314. \IF{2.512}}
		
		\item \longformpublication{Xiong Xiong, \textbf{Lu Hou}, Long Zhao, ``A Group-Based Massive Multiple Access Scheme in Cellular M2M Networks,'' \textit{Computer Communications}, vol. 121, May. 2018, pp. 44-49. \IF{3.338}}
	\end{enumerate}			
	% As an alternative to a long-form publication list, you can create a shorter summary using only DOI values and years.
	
	% Example \doipublication{} command to add another publication:
	
	%\doipublication{Year}{DOI}{firstauthor}{spaceafter}
	
	% All four parameters are required (can be empty though)
	% A value of "firstauthor" in the third parameter will print the DOI in bold
	% A "spaceafter" value in the fourth parameter will add some vertical space -- this is to be used between years
	
	%------------------------------------------------
	
	\medskip % Extra whitespace before the next section
	
	%----------------------------------------------------------------------------------------
	
\end{paracol}

%----------------------------------------------------------------------------------------
\newpage

\begin{paracol}{2} % Begin the multi-column environment

%----------------------------------------------------------------------------------------
%	NAME AND CURRICULUM VITAE TEXT
%----------------------------------------------------------------------------------------

\parbox[top][0.12\textheight][c]{\linewidth}{ % Parbox to hold the author name and CV text; fixed height to match the coloured box to the right, centred vertically and full line width
	\vspace{-0.04\textheight} % Reduce whitespace above the parbox to separate it from the main content
	\centering % Centre text
	{\Huge Lu Hou}\hspace{1pt} \faMars\\\medskip % Your name
	{\Huge\color{headings}\cvtextfont Curriculum Vitae}
}

%----------------------------------------------------------------------------------------
%	MAJOR RESEARCH PROJECT
%----------------------------------------------------------------------------------------

\section{Doctoral Research}

{\raggedright\textbf{``Research on Resource Management and Optimization of Mobile Cloud Networks for Internet of Vehicles" from \textit{The National Natural Science Foundation of China}}\\\medskip}

My research mainly focuses on the resource management issues in multi-layer cloud based Internet of Vehicles (IoV). Aiming at the features of IoV traffic, I proposed some key algorithms for resources optimization of communications, computing and storage with the help convex optimization, machine learning, etc.

\medskip % Extra whitespace before the next section

%----------------------------------------------------------------------------------------
%	WORK EXPERIENCE
%----------------------------------------------------------------------------------------
\section{Major Research Directions}
	\begin{supertabular}{rl}
		\tableentry{Internet of Things Cloud}{Architecture designs, implementations}{}
		\tableentry{}{and performance optimizations.}{spaceafter}
		\tableentry{LoRaWAN\texttrademark \space \space Systems}{Protocol analysis, Architecture designs}{}
		\tableentry{}{implementations and }{}
		\tableentry{}{performance optimizations.}{spaceafter}
		\tableentry{Vehicular Networks}{Resource management of}{}
		\tableentry{}{cloud based vehicular networks}{spaceafter}
	\end{supertabular}

\section{Education} 
\begin{supertabular}{rl} % Start a table with two columns, the table will ensure everything is aligned
	
	%------------------------------------------------
	
	\educationentry{2014 -- 2019} % Duration
	{Doctor of Philosophy} % Degree
	{Communications Engineering} % Honours, achievements or distinctions (e.g. first class honours)
	{Intell. Comput. and Commun. Lab} % Department
	{Beijing Univ. of Posts and Telecom.} % Institution
	
	%------------------------------------------------
	
	\educationentry{2010 -- 2014} % Duration
	{Bachelor of Engineering} % Degree
	{Communications Engineering} % Honours, achievements or distinctions (e.g. first class honours)
	{School of Inform. and Commun. Eng.} % Department
	{Beijing Univ. of Posts and Telecom.} % Institution
	
	%------------------------------------------------
	\educationentry{June 2019} % Duration
	{Graduation}% Degree
	{} % Honours, achievements or distinctions (e.g. first class honours)
	{} % Department
	{Beijing Univ. of Posts and Telecom.} % Institution
\end{supertabular}

%----------------------------------------------------------------------------------------
%	AWARDS
%----------------------------------------------------------------------------------------
\section{Expected position}

\begin{enumerate}[font=\color{headings}\bfseries] % Start a table with two columns, the table will ensure everything is aligned
	
	%------------------------------------------------
	
	\item Engineering or researching position on IoT(such as Smart City).
	
	%------------------------------------------------
	
	\item Python related Development or project management position.
\end{enumerate}

\section{Awards}

% Example \tableentry{} command to add another line:

%\tableentry{Heading}{Content}{spaceafter}

% All 3 parameters must be supplied but any can be empty if you don't need them
% A "spaceafter" value in the third parameter will add some vertical space -- this is to be used between headings

%------------------------------------------------

\begin{supertabular}{rl} % Start a table with two columns, the table will ensure everything is aligned
	
	%------------------------------------------------
	
	\tableentry{2017}{\textbf{China National Scholarship}}{}
	\tableentry{}{\textit{Intell. Comput. and Commun. Lab}}{}
	\tableentry{}{\textit{Beijing Univ. of Posts and Telecom.}}{spaceafter}
	
	%------------------------------------------------
	\tableentry{2018}{\textbf{BUPT Excellent Ph.D. Students Foundation}}{}
	\tableentry{}{\textit{Intell. Comput. and Commun. Lab}}{}
	\tableentry{}{\textit{Beijing Univ. of Posts and Telecom.}}{spaceafter}
	%------------------------------------------------
	
	\tableentry{2018}{\textbf{$2^{th}$ LiangFei Scholarship}}{}
	\tableentry{}{\textit{Intell. Comput. and Commun. Lab}}{}
	\tableentry{}{\textit{Beijing Univ. of Posts and Telecom.}}{spaceafter}
	
	%------------------------------------------------
	
\end{supertabular}

\section{English Skills}

Able to \emph{Listen}, \emph{Speak}, \emph{Read} and \emph{Write} English skillfully.
%----------------------------------------------------------------------------------------
%	REFERENCES
%----------------------------------------------------------------------------------------

%\section{References}
%
%%\textit{References available on request}
%
%%------------------------------------------------
%
%% Example \tableentry{} command to add another line:
%
%%\tableentry{Heading}{Content}{spaceafter}
%
%% All 3 parameters must be supplied but any can be empty if you don't need them
%% A "spaceafter" value in the third parameter will add some vertical space -- this is to be used between headings
%
%%------------------------------------------------
%
%\begin{supertabular}{rl} % Start a table with two columns, the table will ensure everything is aligned
%	
%	%------------------------------------------------
%	
%	\tableentry{}{\textbf{Dr. Isaac Kleiner}}{spaceafter}
%	\tableentry{Position}{Professor}{}
%	\tableentry{Employer}{\href{http://web.mit.edu/physics/}{Department of Physics}}{}
%	\tableentry{}{\href{https://web.mit.edu}{\textit{Massachusetts Institute of Technology}}}{spaceafter}
%	\tableentry{Phone}{+1 (617) 253 1000 x5322 (Work)}{}
%	\tableentry{Mobile}{+1 (232) 842-3583}{}
%	
%	%------------------------------------------------
%	
%	\tableentry{}{}{} % Creates some additional whitespace between the references
%	
%	%------------------------------------------------
%	
%	\tableentry{}{\textbf{Dr. Eli Vance}}{spaceafter}
%	\tableentry{Position}{Scientist (HL1)}{}
%	\tableentry{Employer}{\href{http://www.bmrf.us}{Black Mesa Research Facility}}{spaceafter}
%	\tableentry{Email}{\href{mailto:e.vance@bmrf.us}{e.vance@bmrf.us}}{}
%	\tableentry{Phone}{+1 (800) 786-1410 x6235 (Work)}{}
%	\tableentry{Mobile}{+1 (201) 632-3901}{}
%	
%	%------------------------------------------------
%	
%\end{supertabular}
%
%\medskip % Extra whitespace before the next section

%----------------------------------------------------------------------------------------

\switchcolumn % Switch to the next paracol column

%----------------------------------------------------------------------------------------
%	COLOURED CONTACT DETAILS BOX
%----------------------------------------------------------------------------------------

\parbox[top][0.12\textheight][c]{\linewidth}{ % Parbox to hold the colour box; fixed height to match the name/CV text to the left, centred vertically and full line width
	\vspace{-0.04\textheight} % Reduce whitespace above the parbox to separate it from the main content
	\colorbox{shade}{ % Create the coloured box
		\begin{supertabular}{p{0.05\linewidth}|p{0.775\linewidth}} % Start a table with two columns, the table will ensure everything is aligned
			\raisebox{-1pt}{\faHome} & No. 10, Xitucheng Road, Beijing \\ % Address
			\raisebox{-1pt}{\faPhone} & +86-15501081468 \\ % Phone number
			\raisebox{0pt}{\small\faEnvelope} & \href{mailto:houlu8674@bupt.edu.cn}{houlu8674@bupt.edu.cn} \\ % Email address
			\raisebox{-1pt}{\small\faDesktop} & \href{https://www.houlu.me}{https://www.houlu.me} \\ % Website
			\raisebox{-1pt}{\faGithub} & \href{https://github.com/houluy}{https://github.com/houluy} \\ % GitHub profile
			\raisebox{-1pt}{\faLinkedinSquare} & \href{https://www.linkedin.com/in/username}{https://www.linkedin.com/in/houlu} \\ % LinkedIn profile
			\raisebox{-1pt}{\faWechat} &
			guaguade\\
			\raisebox{-1pt}{\faWeibo} &
			\href{http://weibo.com/lucima}{astroooooo}\\
			% See fontawesome.pdf in the fonts folder for all icons you can use
		\end{supertabular}
	}
}

%----------------------------------------------------------------------------------------
%	EDUCATION
%----------------------------------------------------------------------------------------
\section{Major Project Experiences}

% Blank \workposition command to add another job:

%\workposition{} % Duration
%{} % FT/PT (full time or part time)
%{} % Employer
%{} % Job title
%{} % Description

% All 5 parameters must be supplied but any can be empty if you don't need them

%------------------------------------------------

\workposition{Sep. 2014 -- June 2015} % Duration
{} % FT/PT (full time or part time)
{DSRC protocol stack and applications} % Employer
{\textit{Developer}} % Job title
{Participate in the development of \textit{IEEE 802.11p} protocol stack, and implements some applications such as collision warning on ARM development board. \textbf{Major developing language: C.}} % Description

%------------------------------------------------

\workposition{Sep. 2016 -- Now} % Duration
{} % FT/PT (full time or part time)
{IoT cloud} % Employer
{\textit{Deputy Team Leader}} % Job title
{Responsible for the management of teammates and project schedule, the development of authentication and authorization for servers, access for Amazon\faAmazon Alexa, searching services with Elasticsearch, and the deployment, operations and performance evaluations. \textbf{Major developing language: Node.js}.}  % Description

%------------------------------------------------

\workposition{Oct. 2017 -- Now} % Duration
{} % FT/PT (full time or part time)
{LoRaWAN\texttrademark system} % Employer
{\textit{Team Leader}} % Job title
{Leading the team to design and implement the whole LoRaWAN\texttrademark system for IoT applications, including key algorithms such as Adaptive Data Rate (ADR) controls. As a team leader, I need to manage the whole project schedule, assign tasks to members and make summaries. Besides, I have to handle the technical problems and directions, making sure the project runs on the correct rail. I also need to control the documents, keeping it stick to the projects. I'm responsible for some development in this project. \textbf{Major developing language: Node.js}.} % Description

%------------------------------------------------

\workposition{Sep. 2015 -- Now} % Duration
{} % FT/PT (full time or part time)
{Research on Vehicular Networks} % Employer
{\textit{Individual}} % Job title
{My doctoral research throughout my Ph.D. By using convex optimization, data analyze, machine learning, etc. I tried to study the resource management issues in cloud based IoV. During the study, I've taken part in several NSFC or horizontal projects. \textbf{Major researching language: Python, major tools: IPython, TensorFlow, etc.}} % Description

\vspace{-\baselineskip}\medskip % Standardise the whitespace after this section and before the next (the custom command adds too much otherwise)


% Blank \educationentry{} command to add another degree:

%\educationentry{} % Duration
%{} % Degree
%{} % Honours, achievements or distinctions (e.g. first class honours)
%{} % Department
%{} % Institution

% All 5 parameters must be supplied but any can be empty if you don't need them

%------------------------------------------------



%----------------------------------------------------------------------------------------
%	COMPUTER SKILLS
%----------------------------------------------------------------------------------------

\section{Programming Skills} 

% Example \tableentry{} command to add another line:

%\tableentry{Heading}{Content}{spaceafter}

% All 3 parameters must be supplied but any can be empty if you don't need them
% A "spaceafter" value in the third parameter will add some vertical space -- this is to be used between headings

%------------------------------------------------

\begin{supertabular}{rl} % Start a table with two columns, the table will ensure everything is aligned
	
	\tableentry{Expert}{Python\includegraphics*[width=0.019\textwidth]{python.png}, Node.js\includegraphics*[width=0.019\textwidth]{nodejs.png}}{spaceafter}

	%------------------------------------------------
	
	\tableentry{Intermediate}{C, \CPP, MATLAB, \LaTeX, SQL}{}
	\tableentry{}{shell, HTML, Javascript, CSS}{spaceafter}
	
	%------------------------------------------------
	
	\tableentry{Beginner}{Go, Rust, lisp, Java}{spaceafter}
	
	%------------------------------------------------
	
\end{supertabular}

\section{Computer Skills}
%----------------------------------------------------------------------------------------
%	COMMUNICATION SKILLS
%----------------------------------------------------------------------------------------
	\begin{supertabular}{rl} % Start a table with two columns, the table will ensure everything is aligned
	
	%------------------------------------------------
	
	\tableentry{Expert}{Microsoft Office, Git, reStructuredText,}{}
	\tableentry{}{Markdown, Linux, vim,}{}
	\tableentry{}{HTTP, MQTT, kafka,}{}
	\tableentry{}{ProtocolBuffer, gRPC, TLS, Locust,}{}
	\tableentry{}{Github, StackOverflow, Readthedocs}{}
	\tableentry{}{...}{spaceafter}
	%------------------------------------------------
	
	\tableentry{Intermediate}{Elasticsearch, Nginx, HAProxy, Redis,}{}
	\tableentry{}{MySQL, MongoDB, }{}
	\tableentry{}{...}{spaceafter}
	%------------------------------------------------
	
	%------------------------------------------------
	
\end{supertabular}

% Example \tableentry{} command to add another line:

%\tableentry{Heading}{Content}{spaceafter}

% All 3 parameters must be supplied but any can be empty if you don't need them
% A "spaceafter" value in the third parameter will add some vertical space -- this is to be used between headings

%------------------------------------------------

%----------------------------------------------------------------------------------------
%	SKILLS DESCRIPTION
%----------------------------------------------------------------------------------------

\section{Working Skills}

% Example \longformdescription{} command to add another section:

%\longformdescription{Heading}{Description}

%------------------------------------------------

\longformdescription{Design on service architecture}{Has knowledge on architecture design and practical experience.}

\longformdescription{Developing ability with Python}{Strong ability of Python developing. Running an open techblog on WeChat.}

\longformdescription{Machine learning}{Has knowledge on algorithms of machine learning.}

\longformdescription{Research}{Has the ability of doing researches. Good at discovering and solving problems.}

\longformdescription{Leadership}{Has the ability of leaders.}

\longformdescription{Writing}{Very strong ability of writing documents, reports, records and PPTs.}
%----------------------------------------------------------------------------------------
%	PUBLICATIONS
%----------------------------------------------------------------------------------------
\switchcolumn
\section{Publications}
\begin{enumerate}[font=\color{headings}\bfseries]
	
	\item \longformpublication{\textbf{Lu Hou}, Shaohang Zhao, Xiong Xiong, Kan Zheng, Periklis Chatzimisios, M. Shamim Hossain, Wei Xiang, ``Internet of things cloud: architecture and implementation,'' \textit{IEEE Communications Magazine}, vol. 54, no. 12, Dec. 2016, pp. 32-39. \IF{10.435}}
	
	\item \longformpublication{\textbf{Lu Hou}, Shaohang Zhao, Xing Li, Periklis Chatzimisios, Kan Zheng, ``Design and implementation of application programming interface for Internet of things cloud,'' \textit{International Journal of Network Management}, vol. 27, no. 3, June 2016. \IF{1.118}}
	
	\item \longformpublication{\textbf{Lu Hou}, Kan Zheng, Periklis Chatzimisios, Yi Feng, ``A continuous-time Markov decision process-based resource allocation scheme in vehicular cloud for mobile video services,'' \textit{Computer Communications}, vol. 118, Mar. 2018, pp. 140-147. \IF{3.338}}
	
	\item \longformpublication{\textbf{Lu Hou}, Lei Lei, Kan Zheng, ``Design on publish/subscribe message dissemination for vehicular networks with mobile edge computing,'' \textit{2017 IEEE GLOBECOM}, Dec. 2017}
	
	\item \longformpublication{\textbf{Lu Hou}, Lei Lei, Kan Zheng, Xianbin Wang, ``A Q-learning based Proactive Caching Strategy for Non-safety Related Services in Vehicular Networks,'' \textit{IEEE Internet of Things Journal}, \textbf{Under review}}

	\item \longformpublication{Kan Zheng, \textbf{Lu Hou}, Hanling Meng, Qiang Zheng, Ning Lu, Lei Lei, ``Soft-defined heterogeneous vehicular network: architecture and challenges,'' \textit{IEEE Network}, vol. 30, no. 4, July 2016, pp. 72-80.\IF{7.23}}
	
	\item \longformpublication{Xiong Xiong, \textbf{Lu Hou}, Kan Zheng, Wei Xiang, M. Shamim Hossain, and Sk Md Mizanur Rahman, ``SMDP-based radio resource allocation scheme in software-defined internet of things networks,'' \textit{IEEE Sensors Journal} vol. 16, no. 20, June 2016, pp. 7304-7314. \IF{2.512}}
	
	\item \longformpublication{Xiong Xiong, \textbf{Lu Hou}, Long Zhao, ``A Group-Based Massive Multiple Access Scheme in Cellular M2M Networks,'' \textit{Computer Communications}, vol. 121, May. 2018, pp. 44-49. \IF{3.338}}
\end{enumerate}
% Example \longformdescription{} command to add another publication:

%\longformpublication{Reference (format this manually as desired)}

%------------------------------------------------

%------------------------------------------------

% As an alternative to a long-form publication list, you can create a shorter summary using only DOI values and years.

% Example \doipublication{} command to add another publication:

%\doipublication{Year}{DOI}{firstauthor}{spaceafter}

% All four parameters are required (can be empty though)
% A value of "firstauthor" in the third parameter will print the DOI in bold
% A "spaceafter" value in the fourth parameter will add some vertical space -- this is to be used between years

%------------------------------------------------

\medskip % Extra whitespace before the next section

%----------------------------------------------------------------------------------------

\end{paracol}

%----------------------------------------------------------------------------------------

\end{document}
